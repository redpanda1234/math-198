\documentclass{fkpset}

\name{Forest Kobayashi}
\class{Math 198}
\duedate{01/04/2019}
\assignment{Talk 1}

\chead{Talk 1}
\rhead{Math 198 -- Spring, 2019}
\lfoot{Due Monday, February 4th 2019}



\begin{document}
\begin{problem}[Prompt]
  Prepare and give a 2-3-minute talk that begins with the phrase, ``You wouldn't
  know it by looking at me, but....'' You will not use slides, props, or the
  whiteboard for this talk. Your goal is a give a clear and compelling talk with
  a solid beginning, middle, and end.
\end{problem}
\begin{solution}[Outline.]\setlength{\parindent}{1.5em}
  You wouldn't know it by looking at me, but I have very long toes. Or at least,
  that's what I've been told. Personally, I'm not sure they're \emph{actually}
  long enough to warrant this kind of a big proclamation, but that's probably
  because I'm just used to them, and hence am not well-equipped to judge if
  they're atypical. That's why I'm going to ask for your help here: throughout
  this short talk, I'll give you various descriptions about my toes, and at the
  end, I want \emph{you} to tell \emph{me} whether you think they're strangely
  long. So without further ado, let's jump right in.

  Let me begin by describing feats I can preform with my toes (no pun intended).
  My toes are long enough that I can pick up pencils and things with them, and
  turn doorknobs fairly easily. I can also unscrew caps for some water bottles,
  and even lift them up using my toes, albeit with a high risk of spillage. Of
  course, I'm not sure if this is really atypical --- it could be that everybody
  is capable of such things, but few are silly enough to actually try. Hence,
  let's talk about numbers.

  Last night, I spent some time using a ruler to measure the lengths of each of
  my toes. The results were identical on my left and right feet, so for the
  following I'll just say ``my big toe'' instead of ``my left and right big
  toes.''

  The results: my big toe is 6.0\si{cm}. My index toe is 5.4\si{cm}. My middle
  toe is 4.8\si{cm}, and my ring toe breaks the arithmetic progression, coming
  in at comfortable 4.6\si{cm}. Last but not least, my pinky toe is a quaint
  3.2\si{cm}, just slightly over half of my big toe's length.

  Now, I'm guessing this description is probably less than satisfying to you,
  because those numbers probably don't translate well into a visualization of
  ``size.'' After all: how big even is a centimeter? For the average person,
  it's about the thickness of their pinky finger, but that's just an estimate,
  and we might worry about how much error we'd accumulate in stacking six of
  them together.

  We can circumvent this somewhat by choosing better comparison objects ---
  my big toe is roughly the length of the average human thumb. My index toe is
  very slightly longer than the typical US chicken egg. My middle and ring toes
  are slightly longer than a matchstick, and my pinky toe is almost exactly as
  large as a typical quail egg.

  Ok: this is better, but you're probably still unsure whether you should be
  impressed or not. Are \emph{most} middle toes shorter than a matchstick? Most
  of you probably don't know. What you might \emph{really} want is some
  information the distribution of toe lengths. Questions like ``how many
  standard deviations above the mean are my toe lengths?''

  Boy, I wish I knew. As it turns out, there is almost no data about this
  available online --- and believe me, I looked. This is why I had to
  restructure a lot of my talk

  I guess really, seeing is believing. So go with your gut here: (at this point,
  remove shoes) are my toes weirdly long?

  \begin{enumerate}
    \item Intro
      \begin{itemize}
        \item I have very long toes
        \item \ldots Or so I've been told.
        \item I'm not the best judge
        \item So I'll ask you to help me out
      \end{itemize}
    \item Before numbers, set the stage
      \begin{itemize}
        \item Long enough to pick up pencils and stuff with
        \item Can turn most doorknobs, including smooth ones
        \item Can open some water bottles, and even lift them up
      \end{itemize}
    \item Let's talk numbers
      \begin{itemize}
        \item Measured my toes last night
        \item Brief summary of the methodology
        \item Give lengths
      \end{itemize}
    \item Estimating lengths
      \begin{itemize}
        \item Numbers don't necessarily translate well into notions of size.
        \item Real-world comparisons
      \end{itemize}
    \item Better, but not there yet.
      \begin{itemize}
        \item Still apples to oranges
        \item Hard to confidently assess relative sizing
      \end{itemize}
    \item Apples to apples?
      \begin{itemize}
        \item Can't be done quantitatively (no data)
        \item Leaves only the qualitative option.
      \end{itemize}
  \end{enumerate}

\end{solution}
\end{document}