\documentclass{fkpset}

\name{Forest Kobayashi}
\class{Math 198}
\duedate{01/04/2019}
\assignment{Talk 1}

\chead{Talk 1}
\rhead{Math 198 -- Spring, 2019}
\lfoot{Due Monday, February 4th 2019}



\begin{document}
\begin{problem}[Prompt]
  Prepare and give a 2-3-minute talk that begins with the phrase, ``You wouldn't
  know it by looking at me, but....'' You will not use slides, props, or the
  whiteboard for this talk. Your goal is a give a clear and compelling talk with
  a solid beginning, middle, and end.
\end{problem}
\begin{solution}[Outline.]\setlength{\parindent}{1.5em}
  You wouldn't know it by looking at me, but I have very long toes. Or at least,
  that's what I've been told. Personally, I'm not sure they're \emph{actually}
  long enough to warrant this kind of declaration, but that's probably because
  I'm used to them, and hence am not well-equipped to judge if they're atypical.
  That's why I'm going to ask for your help here: throughout this short talk,
  I'll give you various descriptions about my toes, both quantitative and
  qualitative, and at the end, I want \emph{you} to tell \emph{me} whether
  you think my toes are abnormal or not. So without further ado, let's jump
  right in.

  Let me begin by describing them indirectly. My toes are long enough that I can
  pick up pencils and things with them. I can also pluck a string and turn most
  doorknobs using them. I can even unscrew caps for certain water bottles and
  lift them up using my toes, albeit with a high risk of spillage. But these are
  descriptions of \emph{dexterity}, not length --- and we don't know that they
  necessarily correlate well. And further, we can't be entirely sure that these
  even \emph{prove} atypical nimbleness. It could be that everybody is capable
  of these feats, it's just most people haven't thought to try.

  So let's try a quantitative approach. Last night, I spent some time using a
  ruler to measure the lengths of each of my toes. The results were identical on
  my left and right feet, so when I say ``my big toe is 6\si{cm} long,'' that
  means \emph{both} of my big toes are 6\si{cm} long. I chose to start
  measurement from the end of the crease that separates adjacent toes,
  defaulting to the the inner one when there were two. Finally, measurements
  were taken with my toes relaxed, and did not include nail.

  The results were as follows: my big toe is 6.0\si{cm}. My index toe is
  5.4\si{cm}. My middle toe is 4.8\si{cm}, and my ring toe breaks the arithmetic
  progression, coming in at comfortable 4.6\si{cm}. Last but not least, my pinky
  toe is a quaint 3.2\si{cm}, just slightly over half of my big toe's length.

  Now, I'm guessing this description is probably less than satisfying to you,
  because those numbers probably don't translate well into a visualization of
  ``size.'' After all: how big even is a centimeter? For the average person,
  it's about the thickness of their pinky finger, but that's just an estimate,
  and we might worry about how much error we'd accumulate in stacking six of
  them together.

  We can circumvent this somewhat by choosing better comparison objects ---
  6.5\si{cm} is \emph{about} the length of the average thumb, so my big toe is
  just a bit shorter than that. 5.5\si{cm} is roughly the length of an egg, so
  try to imagine one of those. 4\si{cm} is roughly the length of a matchstick,
  and 3.2 \si{cm} is just about the size of a quail egg, if that helps.

  Ok: this is better, but we're still not there yet. Because even these
  comparisons are flawed. Ignoring concerns about variation in the objects
  themselves, it's still just apples to oranges. I mean, who here can
  confidently assert that their index toe is shorter than a matchstick? You
  probably aren't sure, having never thought to compare the two before. So maybe
  we should just compare toes against toes. Then we could apply tools of
  statistics, and determine \emph{rigorously} whether my toes are, in fact,
  outliers.

  Unfortunately, it turns out data on human toe lengths is sparse. And believe
  me, I tried looking --- I actually spent the better part of half an hour on
  google scholar last night, to no avail. So: what's left? It seems our usual
  tools have failed us --- so how are you to decide?

  I guess really, seeing is believing. So go with your gut here: (at this point,
  remove shoes) are my toes weirdly long?

  \begin{enumerate}
    \item You wouldn't know it by looking at me, but I have very long toes.
    \item
  \end{enumerate}

\end{solution}
\end{document}