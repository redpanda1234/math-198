\documentclass{fkpset}

\name{Forest Kobayashi}
\class{Math 198}
\duedate{01/04/2019}
\assignment{Talk 1}

\chead{Talk 1}
\rhead{Math 198 -- Spring, 2019}
\lfoot{Due Monday, February 4th 2019}



\begin{document}
\begin{problem}[Prompt]
  Prepare and give a 2-3-minute talk that begins with the phrase, ``You wouldn't
  know it by looking at me, but....'' You will not use slides, props, or the
  whiteboard for this talk. Your goal is a give a clear and compelling talk with
  a solid beginning, middle, and end.
\end{problem}
\begin{solution}[Outline.]\setlength{\parindent}{1.5em}
  You wouldn't know it by looking at me, but I have very long toes. Now, before
  I begin telling you some fun facts about ten of my favorite digits, I want to
  address the question that's probably on many of your minds: ``Forest, why the
  heck did you choose this for your talk topic?''

  Allow me to explain. Basically, it boils down to three points: first, 2-3
  minutes is not long enough to build up a really involved personal anecdote, so
  I had to go with something simple and self-contained. Second, I deliberately
  wanted to pick something that's really hard to build a narrative around, since
  I figured it'd be instructive to practice doing so. And third, I just wanted
  to have fun.

  So with that out of the way, let's dig right into it. I hope that right now,
  there's just one question on all of your minds: ``well\ldots just how long
  \emph{are} your toes?'' A natural question. I'll do my best to answer it.

  Last night, I measured the toes on each of my feet separately. Let's talk
  briefly about experimental design here. In my trials, I started measurement
  from the end of that crease thing that separates adjacent toes, and took the
  length from that to the tip, not including nail. For the toes that have
  \emph{two} adjacent creases, I simply chose the inner of the two to use as the
  standard. I didn't want to average the reading from each side, because that
  average corresponded less directly to a length you could \emph{see}. Lastly,
  measurements were taken with my toes relaxed, and hence slightly curled. I did
  not use my hands to straighten them out for the measurements.

  Now for the results. First, for each toe, the measurements were the same on
  both my left and right feet (toes are symmetric across the ``me'' axis, hence
  Noether's theorem implies total podiatric momentum is conserved), so I will
  not refer to things like my ``left big toe'' and ``right big toe'' separately.

  Now, let's talk some numbers: My big toe is 6.0\si{cm}. My index toe, which is
  the next one in (going inwards from medial to lateral), is 5.4\si{cm}. My
  middle toe is --- any guesses here? --- that's right, it's 4.8\si{cm}. My ring
  toe breaks the arithmetic progression here and comes in at comfortable
  4.6\si{cm}. And finally, my pinky toe is a quaint 3.2\si{cm}.

  But maybe we've encountered a problem here: ultimately, those numbers likely
  mean little to any of you. After all: how big even is a centimeter? About the
  width of your pinky finger, maybe --- but what're the error bars on that? What
  does it look like when you stack six of them together? It's hard to tell.
  Despite having been driven towards better visuospatial reasoning by selective
  pressures over the course of many millenia, many humans still struggle a lot
  with standardized length estimation. We can circumvent this problem somewhat
  with the injection of real-world comparisons --- 6.5\si{cm} is \emph{about}
  the length of the average thumb, 5.5\si{cm} is roughly the length of a
  large chicken egg, and 4\si{cm} is about the length of most matchsticks ---
  but there's still something missing here: we don't really know how big toes
  are \emph{supposed} to be.

  For instance, who here can say authoritatively, \emph{without looking}, that
  their index toe is shorter than a matchstick? Before you answer, there's a few
  things you should consider. First, you usually only look at matches when
  you're using them for something, in which case they're much closer to your
  eyes than your toes typically are, so this might bias you to perceive them as
  larger. They're also certainly a lot \emph{narrower} than the average toe, and
  this might contribute to your perception that \emph{length} is their salient
  feature, thus further skewing your judgments.

  It might help instead to take a statistical route. How big are my toes
  relative to the average? Unfortunately, I actually couldn't find any data
  about this online.



  % And then still ---
  % how big even \emph{are} toes usually? What kind of sampling distribution do
  % they follow? Without this information, these numbers still aren't helpful.
  % I completely agree with you.
  % Some of you might know it's roughly the thickness of
  % your pinky finger. But does that overshoot, or \emph{undershoot} on your hand?
  % And what are the error bars like? Maybe you don't remember.


\end{solution}
\end{document}