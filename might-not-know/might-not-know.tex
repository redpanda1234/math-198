\documentclass{fkpset}

\name{Forest Kobayashi}
\class{Math 198}
\duedate{01/04/2019}
\assignment{Talk 1}

\chead{Talk 1}
\rhead{Math 198 -- Spring, 2019}
\lfoot{Due Monday, February 4th 2019}



\begin{document}
\begin{problem}[Prompt]
  Prepare and give a 2-3-minute talk that begins with the phrase, ``You wouldn't
  know it by looking at me, but....'' You will not use slides, props, or the
  whiteboard for this talk. Your goal is a give a clear and compelling talk with
  a solid beginning, middle, and end.
\end{problem}
\begin{solution}[Outline.]\setlength{\parindent}{1.5em}
  You wouldn't know it by looking at me, but I have very long toes. Or at least,
  that's what I've been told. Personally, I'm not sure they're \emph{actually}
  long enough to warrant this kind of declaration, but that's probably because
  I'm used to them, and hence am not well-equipped to judge if they're atypical.
  That's why I'm going to ask for your help here: throughout this short talk,
  I'll give you various descriptions about my toes, both quantitative and
  qualitative, and at the end, I want \emph{you} to tell \emph{me} whether
  you think my toes are abnormal or not. So without further ado, let's jump
  right in.

  Last night, I spent some time measuring the lengths of each of my toes. The
  results were identical on my left and right feet, so when I say ``my big toe
  is 6\si{cm} long,'' that means \emph{both} of my big toes are 6\si{cm} long.
  There are a lots of ways to measure the lengths, but I chose to start
  measurement from the end of the crease that separates adjacent toes,
  defaulting to the the inner one when there were two. Finally, measurements
  were taken with my toes relaxed, and did not include nail.

  The results were as follows: my big toe is 6.0\si{cm}. My index toe is
  5.4\si{cm}. My middle toe is 4.8\si{cm}, and my ring toe breaks the arithmetic
  progression, coming in at comfortable 4.6\si{cm}. Last but not least, my pinky
  toe is a quaint 3.2\si{cm}, just slightly over half of my big toe's length.

  Now, I'm guessing this description is probably less than satisfying to you,
  because ultimately those numbers probably don't translate well into a
  ``size.'' After all: how big even is a centimeter? For the average person,
  it's about the thickness of their pinky finger, but that's just an estimate,
  and we should worry about how much error we'd accumulate in stacking six of
  them together.

  We can circumvent this somewhat by choosing better comparison objects ---
  6.5\si{cm} is \emph{about} the length of the average thumb, so my big toe is
  just a bit shorter than that. 5.5\si{cm} is roughly the length of a large
  chicken egg, so try to imagine one of those where your index toe is and see if
  that seems strange to you, and not just because you have an egg for a toe.
  4\si{cm} is roughly the length of a matchstick, so my middle and ring toes are
  slightly longer than that, and 3.2 \si{cm} is just about the size of a quail
  egg, if that helps.

  Ok: this is better, we're still not there yet. Because even these comparisons
  are flawed. Ignoring concerns about variation in our comparison objects, we're
  still making an apples to oranges comparison. After all, who here can
  authoritatively say that their index toe is shorter than a matchstick? You've
  probably never thought to compare the two before, and so aren't sure.

  % Before you answer, consider that you usually only look at matches when you're
  % holding them, in which case they're much closer to your eyes than your toes
  % typically are. This might bias you to perceive them as larger. And also,
  % matchsticks are certainly a lot \emph{narrower} than the average toe, hence
  % you might implicitly identify \emph{length} as their salient feature.

  It's for situations like these that we would like to fall back on statistical
  tools. We don't have good intuition for whether or not index toes typically
  \emph{are} shorter or longer than a matchstick, so we make a comparison of
  toes to toes. In other words, we want to obtain data about the
  \emph{distribution} of standard toe lengths if we want to definitively
  identify me as an outlier. But unfortunately, when I googled around for human
  toe statistics, I couldn't find anything!

  So, it seems like all of our usual analytical tools have failed us here. At
  the beginning

  I
  guess in this situation,



  \begin{enumerate}
    \item You wouldn't know it by looking at me, but I have very long toes.
    \item
  \end{enumerate}

\end{solution}
\end{document}