\documentclass{fkpset}

\usepackage{hyperref}

\name{Forest Kobayashi}
\class{Math 198}
\duedate{1/28/2019}
\assignment{Ted Talk Exercise}

\chead{Math Forum}
\rhead{Spring -- 2019}

\begin{document}
\begin{problem}[Exercise]
  Watch a TED talk in groups of 3-4 and answer the following questions (to
  submit individually although 1-6 can be the same for your group):
  \begin{enumerate}[label=\arabic*.]
    \item What was the topic/title of the talk?
    \item Give the URL of the talk
    \item List 5 effective rhetorical moves made by the speaker (if familiar w/
      R.M.)
    \item Comment on the motion, facial expressions and body language of the
      speaker
    \item What was the least successful aspect of the talk?
    \item How was the talk organized? Create a simple outline of the talk that
      communicates its overall structure.
    \item List 5 aspects of the talk that you might try to emulate in your talks
      this semester.
  \end{enumerate}
\end{problem}
\begin{solution}[Answers.]
  \begin{enumerate}[label=\arabic*.]
    \item \emph{Strange Answers to the psychopath test}, by Jon Ronson
      (TED2012).
    \item \url{https://www.ted.com/talks/jon_ronson_strange_answers_to_the_psychopath_test?utm_campaign=tedspread&utm_medium=referral&utm_source=tedcomshare}
    \item Rhetorical moves
      \begin{enumerate}
        \item Speaker does reenactment of conversations vs. just reporting them
          to the audience, which makes the audience feel more engaged
        \item The speaker uses a personal anecdote as a hook in the beginning to
          draw the audience in
        \item Dual to the point above, the speaker finishes the talk by giving
          the audience a question that makes them think about how \emph{they}
          would act in the particular situation posed.
        \item The speaker keeps language colloquial and chatty, to make it
          easier for the audience to relate themselves to the situations being
          presented.
        \item Additionally, the speaker poses many rhetorical questions to the
          audience members, conceivably for similar reasons.
        \item Speech is very metered and evenly-paced.
      \end{enumerate}
    \item Motion, facial expressions, and other body language
      \begin{enumerate}
        \item Uses hand motions to indicate sizes of objects. In fact, the
          speark really uses hand motions extensively to mirror anything he's
          saying that can be physically expressed. E.g., he would raise his
          hands when talking about something ``high'', and lower them when
          talking about something ``low''. Also uses hands to mirror the cadence
          of his speech sometimes.
        \item The speaker uses animations in the background (and sound effects)
          to reinforce what he's saying
        \item Speaker would smile and/or give some facial expression cue for any
          points that were intended to be comical.
      \end{enumerate}
    \item Flaws
      \begin{enumerate}
        \item It was kind of hard to see what the takeaway was supposed to be
          until the end. Even then, we weren't really sure what he was trying to
          accomplish --- comedy made it difficult to discern what was supposed
          to be taken seriously.
        \item The speaker's talk was very narrative-heavy, and it was sometimes
          hard to discern how each piece fit into the broader overarching
          structure of the talk. Some metaphorical mileposts along the way would
          have been very helpful in trying to figure out what we should vs.\
          shouldn't be paying attention to.
        \item Without more overarching structure to tie it together, repeatedly
          switching between two or three story threads became confusing.
      \end{enumerate}
    \item Loose outline of the structure
      \begin{enumerate}
        \item Introduction
          \begin{enumerate}[label=\roman*.]
            \item Stuff about the DSM and his own personal anecdote establishing
              a connection to the field of psychiatry
            \item Introduces the question of whether or not these checklists are
              truly meaningful/effective at identifying whether people have
              mental illnesses
            \item Transitions to an extended anecdote
          \end{enumerate}
        \item Anecdote from Tony about Broadmoor
        \item Viewpoints of experts
        \item Chainsaw Al anecdote
        \item More about Tony
        \item Conclusion
          \begin{enumerate}[label=\roman*.]
            \item Big picture: gray areas and the humanity reflected in them
            \item Closes by posing a question to the audience.
          \end{enumerate}
      \end{enumerate}
    \item 5 aspects I might want to emulate
      \begin{enumerate}
        \item Speaker talks colloquially for almost the entire talk; makes the
          audience member feel like the material is accessible and relatable.
        \item Speaker does a good job of creating a sense of narrative
          throughout the whole talk, even when discussing technical bits.
        \item Speaker frames the closing remarks in such a way that listeners
          continue to think and/or wonder about the material, even once the talk
          is over.
        \item Speaker replaces pacing (which is often distracting) with body
          language like hand movements (much less so). Similarly, instead of
          turning with his whole body to address different parts of the
          audience, the speaker looks around mainly with his head/neck, which is
          also less distracting.
        \item The speakers' cadence was very level and consistent, which made
          the talk sound less ``theatrical'' than most TED talks I've seen,
          which made it easier to engage as an audience member (I didn't feel
          like a certain response was ``expected'' of me, which made it easier
          to figure out how I personally felt about the talk).
      \end{enumerate}
  \end{enumerate}
\end{solution}
\end{document}