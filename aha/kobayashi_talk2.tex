\documentclass{fkpresentation}

% Information to be included in the title page:
\title{\textmd{A ``Metaphorical'' Theorem}}
\author{Forest Kobayashi}
\institute{Harvey Mudd College}
\date{February 11th, 2018}


\usetheme{Dresden}
\usecolortheme{crane}
% Make the colors look halfway passable, at least for now.
% \definecolor{lgray}{RGB}{230, 230, 230}
% \definecolor{tgray}{RGB}{70, 70, 70}
% \definecolor{mgold}{RGB}{238, 204, 119}
% \mode<presentation> {
% \usetheme{Dresden}
% \setbeamercolor*{palette secondary}{use=structure,fg=tgray,bg=lgray}
% \setbeamercolor*{palette tertiary}{use=structure,fg=lgray,bg=tgray}
% \setbeamercolor*{normal text}{fg=darkgray}
% \setbeamercolor*{frametitle}{fg=black}
% \setbeamercolor*{title}{fg=black}
% \setbeamerfont*{section in head/foot}{size=\scriptsize}
% \setbeamerfont*{subsection in head/foot}{size=\scriptsize}
% }
% Partial differential
\newcommand{\pdd}{\mathop{}\,\partial}

% Total differential
\newcommand{\dd}{\mathop{}\,\mathrm{d}}

% Define partial derivative and ordinary derivative things
\newcommand{\pd}[3][]{\frac{\partial^{#1}\hspace{-.08em} {#2}}{\partial {#3}^{#1}}}
\newcommand{\od}[3][]{\frac{{d}^{#1}\hspace{-.08em} {#2}}{d {#3}^{#1}}}

\usetikzlibrary{quotes,arrows.meta}
\tikzset{
  annotated cuboid/.pic={
    \tikzset{%
      every edge quotes/.append style={midway, auto},
      /cuboid/.cd,
      #1
    }
    \draw [every edge/.append style={pic actions, densely dashed, opacity=.5}, pic actions]
    (0,0,0) coordinate (o) -- ++(-\cubescale*\cubex,0,0) coordinate (a) -- ++(0,-\cubescale*\cubey,0) coordinate (b) edge coordinate [pos=1] (g) ++(0,0,-\cubescale*\cubez)  -- ++(\cubescale*\cubex,0,0) coordinate (c) -- cycle
    (o) -- ++(0,0,-\cubescale*\cubez) coordinate (d) -- ++(0,-\cubescale*\cubey,0) coordinate (e) edge (g) -- (c) -- cycle
    (o) -- (a) -- ++(0,0,-\cubescale*\cubez) coordinate (f) edge (g) -- (d) -- cycle;
    \path [every edge/.append style={pic actions, |-|}]
    (b) +(0,-5pt) coordinate (b1) edge ["\cubex \cubeunits"'] (b1 -| c)
    (b) +(-5pt,0) coordinate (b2) edge ["\cubey \cubeunits"] (b2 |- a)
    (c) +(3.5pt,-3.5pt) coordinate (c2) edge ["\cubez \cubeunits"'] ([xshift=3.5pt,yshift=-3.5pt]e)
    ;
  },
  /cuboid/.search also={/tikz},
  /cuboid/.cd,
  width/.store in=\cubex,
  height/.store in=\cubey,
  depth/.store in=\cubez,
  units/.store in=\cubeunits,
  scale/.store in=\cubescale,
  width=10,
  height=10,
  depth=10,
  units=cm,
  scale=.1,
}

\usepackage[style=british]{csquotes}

\def\signed #1{{\leavevmode\unskip\nobreak\hfil\penalty50\hskip1em
    \hbox{}\nobreak\hfill #1%
    \parfillskip=0pt \finalhyphendemerits=0 \endgraf}}

\newsavebox\mybox
\newenvironment{aquote}[1]
{\savebox\mybox{#1}\begin{quote}\openautoquote\hspace*{-.7ex}}
  {\unskip\closeautoquote\vspace*{1mm}\signed{\usebox\mybox}\end{quote}}


\theoremstyle{plain}% default
\newtheorem{proposition}[theorem]{Proposition}

\begin{document}
\frame{\titlepage}
\section{Introduction}
\begin{frame}{Today's talk}
  \begin{itemize}
    \item Prompt: ``focus your talk on a favorite mathematical idea or
      theorem.'' \pause
    \item Original presentation: interplay between visuals and algebra
      \begin{itemize}
        \item Technical
        \item Not what I \emph{really} wanted to talk about\pause
      \end{itemize}
    \item Prompt, redux: ``A \emph{\color{red}personal} `Aha!' moment in
      mathematics.''\pause
      \begin{itemize}
        \item Yeah, I can do that.
      \end{itemize}
  \end{itemize}
\end{frame}
\begin{frame}{Agenda:}
  Today, we'll talk about perseverance. \pause
  \begin{theorem}
    Hard work pays off.
  \end{theorem}
\end{frame}

\section{Setting the stage}
\begin{frame}{Backstory}
  \begin{itemize}
    \item Middle school
    \item High school
      \begin{itemize}
        \item Math: by \emph{far} my worst subject
        \item Ms.\ Bender's Precalc class
        \item 11\textsuperscript{th} grade --- almost kicked out of the honors
          program
      \end{itemize}
  \end{itemize}
\end{frame}

\section{Proof by Anecdote}

\section{Conclusion}
\begin{frame}{The general case:}
  \vfill
  \begin{theorem}
    Hard work pays off.
  \end{theorem}\pause
  \begin{proof}
    Left as an exercise.
  \end{proof}
  \vfill
\end{frame}

\end{document}


%%% Local Variables:
%%% mode: latex
%%% TeX-master: t
%%% End